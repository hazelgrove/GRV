\documentclass[letterpaper,12pt]{article}

\title{GRV Formal Syntax and Operational Semantics}
\author{Eric Griffis \\ egriffis@umich.edu}

% Page Layout

\usepackage[margin=1in]{geometry}

\usepackage{parskip}
\setlength{\parindent}{0pt}

% Content Layout

\usepackage{array}
\newcolumntype{C}{>{$}c<{$}}
\newcolumntype{L}{>{$}l<{$}}
\newcolumntype{R}{>{$}r<{$\ \:\!}}

\newenvironment{Grammar}
{
  \begin{tabular*}{\textwidth}{
    >{$}l<{$}
    >{$}c<{$}
    >{$}l<{$}
    @{\extracolsep{\fill}}
    r}
}{
  \end{tabular*}
}

\newcommand\OR{\ensuremath{~|~}}

\newenvironment{alignRL}{%
\begingroup
\setlength{\tabcolsep}{0pt}
\begin{tabular}{RL}%
  }{%
\end{tabular}
\endgroup%
}

% Notation

\usepackage{amsmath,amssymb}
\usepackage{semantic}
\usepackage{colonequals}

\mathlig{::=}{\coloncolonequals}
\mathlig{|-}{\vdash}
\mathlig{:->}{\mapsto}
\mathlig{-->a}{\overset{a}{\longrightarrow}}
\mathlig{:}{\!:\!}

\def\Cup{\cup\ \:\!}

% Inference Rules

\usepackage{mathpartir}

\newcommand\Rule[1]{\text{rule \RULE{#1}}}
\newcommand\RULE[1]{\text{\textsc{#1}}}
\newcommand\Inferrule[3][]{\inferrule{#2}{#3}~\RULE{#1}}

\def\A{\mathcal{A}}
\def\C{\mathcal{C}}
\def\E{\mathcal{E}}
\def\G{\mathcal{G}}
\def\I{\mathcal{I}}
\def\K{\mathcal{K}}
\def\S{\mathcal{S}}
\def\U{\mathcal{U}}
\def\V{\mathcal{V}}

\def\GG{\text{G}}

\def\e{\varepsilon}

\def\Create{\text{Create}}
\def\Destroy{\text{Destroy}}
\def\Down{\text{Down}}
\def\Enqueue{\text{Enqueue}}
\def\Left{\text{Left}}
\def\Move{\text{Move}}
\def\Num{\text{Num}}
\def\Right{\text{Right}}
\def\Select{\text{Select}}
\def\Send{\text{Send}}
\def\Up{\text{Up}}

\DeclareMathOperator{\children}{\text{children}}
\DeclareMathOperator{\domain}{\text{domain}}
\DeclareMathOperator{\parents}{\text{parents}}

\def\None{\varnothing}
\DeclareMathOperator{\publish}{\text{publish}}
\DeclareMathOperator{\leftIndex}{\text{leftIndex}}
\DeclareMathOperator{\rightIndex}{\text{rightIndex}}
\DeclareMathOperator{\downIndex}{\text{downIndex}}
\DeclareMathOperator{\defaultIndex}{\text{defaultIndex}}

%%%%%%%%%%%%%%%%%%%%%%%%%%%%%%%%%%%%%%%%%%%%%%%%%%%%%%%%%%%%%%%%%%%%%%%%%%%%%%%%

\begin{document}

\maketitle

% ==============================================================================

\section{Formal Syntax}
\label{sec:formal-syntax}

\begin{minipage}[t]{0.5\textwidth}
  \vspace{0pt}
  \begin{Grammar}
    e
    & ::= & \lambda p : \tau.e & abstraction \\
    & \OR & e~e                & application \\
    & \OR & e+e                & addition    \\
    & \OR & n                  & number      \\
    \\
    p
    & ::= & x & variable pattern \\
    \\
    \tau
    & ::= & \tau -> \tau & function type \\
    & \OR & \Num         & number type   \\
  \end{Grammar}
\end{minipage}

% ==============================================================================

\section{Concepts}
\label{sec:concepts}

\begin{tabular}{cl@{\hspace{1.5cm}}cl@{\hspace{1.5cm}}cl}
  $\A$ & Actions    & $\G$ & Graphs          & $\S$ & States   \\
  $\C$ & Cursors    & $\I$ & Indices         & $\U$ & UUIDs    \\
  $\E$ & Edges      & $\K$ & Constructors    & $\V$ & Vertices \\
\end{tabular}

% ------------------------------------------------------------------------------

\subsection{Unique IDs}
\label{sec:unique-ids}

$u \in \U$

% ------------------------------------------------------------------------------

\subsection{The Target Language}
\label{sec:the-target-language}

A \emph{constructor} represents a node in the target language's abstract
syntax tree.

$k \in \K$

An \emph{index} names a potential child vertex.

$i \in \I$

% ------------------------------------------------------------------------------

\subsection{The Graph}
\label{sec:the-graph}

% ..............................................................................

\subsubsection{Vertex}
\label{sec:vertex}

A \emph{vertex} is a uniquely identifiable instance of a constructor.

$\V = \U \times \K$

$v \in \V$

$v = (u, k)$

% ..............................................................................

\subsubsection{Cursor}
\label{sec:cursor}

A \emph{cursor} is a ``soft'' link---the target vertex is referenced by index
and may not exist.

$\C$ = $\V \times \I$

$c \in \C$

$c = (v, i)$

% ..............................................................................

\subsubsection{Edge}
\label{sec:edge}

An \emph{edge} is a ``hard'' link---the target vertex must exist.

$\E = \C \times \V$

$\e \in \E$

$\e = (c, v)$

% ..............................................................................

\subsubsection{State}
\label{sec:state}

The \emph{state} of an edge determines whether it has been created or
destroyed.

$\S = \{\bot,+,-\}$

$s \in \S$

% ..............................................................................

\subsubsection{Graph}
\label{sec:graph}

A \emph{graph} is a map from edges to states.

$\G = \bigcup\limits_{n \ge 0}\G^n = \bigcup\limits_{n \ge 0}(\E \times \S)^n$

$\Gamma \in \G$

$\Gamma = \{\e_1 :-> s_1, \e_2 :-> s_2, \ldots, \e_n :-> s_n\} = \{\e_k :-> s_k\}_{k=1}^n$

$\Gamma(\e) = \{s_k\}_{k=1}^n \iff \{\e :-> s_k\}_{k=1}^n \subseteq \Gamma
\land (\Gamma \setminus \{\e :-> s_k\}_{k=1}^n)(\e) = \None$

$\forall \Gamma, \exists n \in \mathbb{N} \,:\, \Gamma \in \G^n$

$| \Gamma | = n \iff \Gamma \in \G^n$

% ------------------------------------------------------------------------------

\subsection{The Editor}
\label{sec:the-editor}

An \emph{editor} models a graph that evolves according to a stream of actions.

$E = \G \times \C \times \A^{\mathbb{N}} \times \E^{\mathbb{N}}$

% ..............................................................................

\subsubsection{Actions}
\label{sec:actions}

An \emph{action} describes a change to one or more editors.

\begin{alignRL}
  \A &= \A^{move} \cup \A^{edit} \cup \A^{ctrl} \\
  a &\in \A
\end{alignRL}

\paragraph{Move actions} reposition the cursor.

\begin{alignRL}
  \A^{move}
  &= \{ \Left,\Right,\Up,\Down \} \\
  &\Cup \{ \Select \} \! \times \C
\end{alignRL}

\paragraph{Edit actions} add or remove an edge at the cursor.

\begin{alignRL}
  \A^{edit}
  &= \{ \Create \} \times \K \\
  &\Cup \{ \Destroy \}
\end{alignRL}

\paragraph{Control actions} replicate actions from one context to another.

$\A^{ctrl} = \{ \Send \} \times (\A^{edit})^{\mathbb{N}}$

% ==============================================================================

\section{Operational Semantics}
\label{sec:operational-semantics}

\begin{alignRL}
   \parents(\Gamma,v) &= \{ \e_i=(c_i,v_i) \,:\, \Gamma(\e_i)=+, v_i=v \} \\
  \children(\Gamma,c) &= \{ \e_i=(c_i,v_i) \,:\, \Gamma(\e_i)=+, c_i=c \}
\end{alignRL}

\vspace*{2ex}

$\boxed{\Gamma,c,\e^{*},a^{*} |- a --> \Gamma,c,\e^{*},a^{*}}$

\begin{mathpar}
  \Inferrule{
    a \in \A^{edit} \\
    \Gamma,c,\e^{*} -->a \Gamma',c',\e'^{*}
  }{
    \Gamma,c,\e^{*},a^{*} |- a --> \Gamma',c',\e'^{*},a^{*}a
  }

  \Inferrule{
    a \notin \A^{edit} \\
    \Gamma,c,\e^{*} -->a \Gamma',c',\e'^{*}
  }{
    \Gamma,c,\e^{*},a^{*} |- a --> \Gamma',c',\e'^{*},a^{*}
  }

  \Inferrule{
    \publish(a'^{*}) = ()
  }{
    \Gamma,c,\e^{*},a^{*} |-  \Send~a'^{*} -->
    \Gamma,c,\e^{*},a^{*} \setminus a'^{*}
  }
\end{mathpar}
 
$\boxed{\Gamma,c,\e^{*} -->a \Gamma,c,\e^{*}}$

\vspace*{1ex}

\begin{mathpar}
  \Inferrule{
    \leftIndex(i) = i'
  }{
    \Gamma,(v,i),\e^{*} \xrightarrow{\Left} \Gamma,(v,i'),\e^{*}
  }

  \Inferrule{
    \rightIndex(i) = i'
  }{
    \Gamma,(v,i),\e^{*} \xrightarrow{\Right} \Gamma,(v,i'),\e^{*}
  }

  \Inferrule{
    \parents(v) = (c',v')
  }{
    \Gamma,(v,i),\e^{*} \xrightarrow{\Up} \Gamma,c',\e^{*}
  }

  \Inferrule{
    \children(c) = (c',v') \\
    v' = (u,k) \\
    \downIndex(k) = i
  }{
    \Gamma,c,\e^{*} \xrightarrow{\Down} \Gamma,(v',i),\e^{*}
  }

  \Inferrule{}{
    \Gamma,c,\e^{*} \xrightarrow{\Select~c'} \Gamma,c',\e^{*}
  }

  \Inferrule{
    \defaultIndex(c) = \None \\
    v = (u,k) \\
    u \in \U \text{ fresh} \\
    \e = (c,v) \\
    \Gamma(\e) \ne -
  }{
    \Gamma,c,\e^{*} \xrightarrow{\Create~k}
    \Gamma[\e :-> +],c,\e^{*}
  }

  \Inferrule{
    \defaultIndex(c) = i' \\
    v = (u,k) \\
    u \in \U \text{ fresh} \\
    \e = (c,v) \\
    \Gamma(\e) \ne -
    \\\\
    \children(\Gamma,c) = \{\e_k\}_{k=1}^n = \{(c_k,v_k)\}_{k=1}^n \\
    c' = (v,i')
  }{
    \Gamma,c,\e^{*} \xrightarrow{\Create~k}
    \Gamma[\e :-> +][(c',v_k) :-> +]_{k=1}^n[\e_k :-> -]_{k=1}^n,c',\e^{*}
  }

  \Inferrule{
    \children(\Gamma,c) = \{\e_k\}_{k=1}^n \\
    c = (\e,v)
  }{
    \Gamma,c,\e^{*} \xrightarrow{\Destroy} \Gamma[\e_k :-> -]_{k=1}^n,c,\e^{*}\e
  }
\end{mathpar}

\end{document}
