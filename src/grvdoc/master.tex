\documentclass[letterpaper,12pt]{article}

\title{GRV Formal Syntax and Operational Semantics}
\author{Eric Griffis \\ egriffis@umich.edu}

% Page Layout

\usepackage[margin=1in]{geometry}

\usepackage{parskip}
\setlength{\parindent}{0pt}

% Content Layout

\usepackage{array}
\newcolumntype{C}{>{$}c<{$}}
\newcolumntype{L}{>{$}l<{$}}
\newcolumntype{R}{>{$}r<{$\ \:\!}}

\newenvironment{Grammar}
{
  \begin{tabular*}{\textwidth}{
    >{$}l<{$}
    >{$}c<{$}
    >{$}l<{$}
    @{\extracolsep{\fill}}
    r}
}{
  \end{tabular*}
}

\newcommand\OR{\ensuremath{~|~}}

\newenvironment{alignRL}{%
\begingroup
\setlength{\tabcolsep}{0pt}
\begin{tabular}{RL}%
  }{%
\end{tabular}
\endgroup%
}

% Notation

\usepackage{amsmath,amssymb}
\usepackage{semantic}
\usepackage{colonequals}

\mathlig{::=}{\coloncolonequals}
\mathlig{|-}{\vdash}
\mathlig{:->}{\mapsto}
\mathlig{-->a}{\overset{a}{\longrightarrow}}
\mathlig{:~}{:}
\mathlig{:}{\!:\!}

\def\Cup{\cup\ \:\!}

% Inference Rules

\usepackage{mathpartir}

\newcommand\Rule[1]{\text{rule \RULE{#1}}}
\newcommand\RULE[1]{\text{\textsc{#1}}}
\newcommand\Inferrule[3][]{\inferrule{#2}{#3}~\RULE{#1}}

\def\A{\mathcal{A}}
\def\C{\mathcal{C}}
\def\E{\mathcal{E}}
\def\G{\mathcal{G}}
\def\I{\mathcal{I}}
\def\K{\mathcal{K}}
\def\S{\mathcal{S}}
\def\U{\mathcal{U}}
\def\V{\mathcal{V}}

\def\e{\varepsilon}
\def\Abf{\textbf{A}}
\def\Ebf{\textbf{E}}

\def\Create{\text{Create}}
\def\Destroy{\text{Destroy}}
\def\Down{\text{Down}}
\def\Enqueue{\text{Enqueue}}
\def\Left{\text{Left}}
\def\Move{\text{Move}}
\def\Num{\text{Num}}
\def\Restore{\text{Restore}}
\def\Right{\text{Right}}
\def\Select{\text{Select}}
\def\Send{\text{Send}}
\def\Up{\text{Up}}

\DeclareMathOperator{\children}{\text{children}}
\DeclareMathOperator{\deleted}{\text{deleted}}
\DeclareMathOperator{\domain}{\text{domain}}
\DeclareMathOperator{\known}{\text{known}}
\DeclareMathOperator{\orphans}{\text{orphans}}
\DeclareMathOperator{\parents}{\text{parents}}
\DeclareMathOperator{\seen}{\text{seen}}
\DeclareMathOperator{\unseen}{\text{unseen}}
\DeclareMathOperator{\vertexes}{\text{vertexes}}

\def\None{\varnothing}
\DeclareMathOperator{\publish}{\text{publish}}
\DeclareMathOperator{\leftIndex}{\text{leftIndex}}
\DeclareMathOperator{\rightIndex}{\text{rightIndex}}
\DeclareMathOperator{\downIndex}{\text{downIndex}}
\DeclareMathOperator{\defaultIndex}{\text{defaultIndex}}

%%%%%%%%%%%%%%%%%%%%%%%%%%%%%%%%%%%%%%%%%%%%%%%%%%%%%%%%%%%%%%%%%%%%%%%%%%%%%%%%

\begin{document}

\maketitle

% ==============================================================================

\section{Formal Syntax}
\label{sec:formal-syntax}

\begin{minipage}[t]{0.5\textwidth}
  \vspace{0pt}
  \begin{Grammar}
    e
    & ::= & \lambda p : \tau.e & abstraction \\
    & \OR & e~e                & application \\
    & \OR & e+e                & addition    \\
    & \OR & n                  & number      \\
    \\
    p
    & ::= & x & variable pattern \\
    \\
    \tau
    & ::= & \tau -> \tau & function type \\
    & \OR & \Num         & number type   \\
  \end{Grammar}
\end{minipage}

% ==============================================================================

\section{Concepts}
\label{sec:concepts}

\begin{tabular}{cl@{\hspace{1.5cm}}cl@{\hspace{1.5cm}}cl@{\hspace{1.5cm}}cl}
  $\A$ & Actions & $\G$ & Graphs       & $\S$ & Edge States & $\Abf$ & Graph Actions \\
  $\C$ & Cursors & $\I$ & Indices      & $\U$ & UUIDs       & $\Ebf$ & Editors       \\
  $\E$ & Edges   & $\K$ & Constructors & $\V$ & Vertexes \\
\end{tabular}

A \emph{uuid} $u \in \U$ is a unique identifier.

A \emph{constructor} $k \in \K$ represents a node in the target language's
abstract syntax tree.

An \emph{index} $i \in \I$ names a position, relative to some parent
constructor, of a potential child node in the target language's AST.

% ------------------------------------------------------------------------------

\subsection{The Graph}
\label{sec:the-graph}

% ..............................................................................

\paragraph{Vertex} ~

$\V = \K \times \U$

A \emph{vertex} $v = (k, u) \in \V$ is an instance of constructor $k$ with
uuid $u$.

% ..............................................................................

\paragraph{Cursor} ~

$\C = \V \times \I$

A \emph{cursor} is a reference to the (possibly empty) set of all vertexes
with parent vertex $v$ and child index $i$.

% ..............................................................................

\paragraph{Edge} ~

$\E = \C \times \V \times \U$

An \emph{edge} $\e = (c, v, u) \in \E$ connects a cursor $c$ to a specific
vertex $v$.

% ..............................................................................

\paragraph{Edge State} ~

$\S = \{\bot,+,-\}$

An \emph{edge state} $s \in \S$ determines whether an edge has been created
and, if so, whether it has been destroyed.

\paragraph{Graph Action} ~

$\Abf = \E \times \S \times \U$

A \emph{graph action} $A = (\e,s,u) \in \Abf$ is an instance of a binding from
edge $\e$ to edge state $s$. Denote by $[\e :-> s]$ a fresh binding from edge
$\e$ to edge state $s$.

\paragraph{Graph} ~

$\G = \Abf^{*}$

A \emph{graph} $(G : \E -> \S) = A^{*} \in \G$ is a function from edges to
edge states modeled as a set of graph actions $A^{*}$. Denote by $G(\e) = s$
the state $s$ of edge $\e$ in graph $G$, and by $G A$ the graph $G$ extended
with graph action $A$.

% ------------------------------------------------------------------------------

\subsection{The Editor}
\label{sec:the-editor}

$\Ebf = \G \times \C \times \A^{*} \times \Abf^{*} \times \U$

An \emph{editor} $E = (G,c,a^{*},A^{*},u) \in \Ebf$ is an instance of graph
$G$ with cursor $c$, action queue $a^{*}$, and known graph actions $A^{*}$.

% ..............................................................................

\paragraph{Actions} ~

$\A = \A^{move} \cup \A^{edit} \cup \A^{comm}$

An \emph{action} $a \in \A$ describes a change to one or more editors.

\paragraph{Move Actions} ~

\begin{alignRL}
  \A^{move}
  &= \{ \Left,\Right,\Up,\Down \} \\
  &\Cup \{ \Select \} \! \times \C
\end{alignRL}

A \emph{move action} $a \in \A^{move}$ repositions the cursor.

\paragraph{Edit Actions} ~

\begin{alignRL}
  \A^{edit}
  &= \{ \Create \} \times \K \\
  &\Cup \{ \Destroy \} \\
  &\Cup \{ \Restore \} \times \V
\end{alignRL}

An \emph{edit action} $a \in \A^{edit}$ adds or removes an edge at the cursor.

\paragraph{Communication Actions} ~

$\A^{comm} = \{ \Send \} \times (\A^{edit})^{*}$

A \emph{communication action} $a \in \A^{comm}$ copies edit actions from one
editor to the others.

% ------------------------------------------------------------------------------

\subsection{The Environment}
\label{sec:the-environment}

An \emph{environment} $E^{*} \in \Ebf^{*}$ is a set of communicating editors.

% ==============================================================================

\section{Operational Semantics}
\label{sec:operational-semantics}

\begin{alignRL}
  \children(G,c) &= \{ \e=(c',v,u) :~ c=c', G(\e)=+ \} \\
  \parents(G,v) &= \{ \e=(c,v',u) :~ v=v', G(\e)=+ \}
\end{alignRL}

\paragraph{Deleted Vertexes} ~

\begin{alignRL}
  \deleted(G) &= \orphans(G) \cup \unseen(G) \\
  \orphans(G) &= \{ v :~ \parents(G,v) = \None \} \\
  \unseen(G) &= \vertexes(G) \setminus \orphans(G) \setminus \seen(G) \\
  \seen(G) &= \{ v :~ \parents(G,v) \subseteq \orphans(G) \cup \seen(G) \}
\end{alignRL}

\vspace*{2ex}

\subsection{Movement}
\label{sec:movement}

$\boxed{E -->a E}$
%
\begin{mathpar}
  \Inferrule{
    \leftIndex(i) = i'
  }{
    G,(v,i),a^{*},A^{*},u \xrightarrow{\Left} G,(v,i'),a^{*},A^{*},u
  }

  \Inferrule{
    \rightIndex(i) = i'
  }{
    G,(v,i),a^{*},A^{*},u \xrightarrow{\Right} G,(v,i'),a^{*},A^{*},u
  }

  \Inferrule{
    \parents(v) = \{(c',v',u')\}
  }{
    G,(v,i),a^{*},A^{*},u \xrightarrow{\Up} G,c',a^{*},A^{*},u
  }

  \Inferrule{
    \children(c) = \{(c',(k,u_k),u')\} \\
    \downIndex(k) = i
  }{
    G,c,a^{*},A^{*},u \xrightarrow{\Down} G,((k,u_k),i),a^{*},A^{*},u
  }

  \Inferrule{}{
    G,c,a^{*},A^{*},u \xrightarrow{\Select~c'} G,c',a^{*},A^{*},u
  }
\end{mathpar}

\subsection{Editing}
\label{sec:editing}

\begin{mathpar}
  \Inferrule{
    \defaultIndex(c) = \None \\
    u, u_k \in \U \text{ fresh} \\
    A = [(c,(k,u_k),u) :-> +]
  }{
    G,c,a^{*},A^{*},u \xrightarrow{\Create~k} G A,c,a^{*},A^{*}A,u
  }

  \Inferrule{
    \defaultIndex(c) = i' \\
    c' = ((k,u_k),i') \\
    u, u_k \in \U \text{ fresh} \\
    \\\\
    \children(G,c) = \{(c_j,v_j,u_j)\}_{j=1}^n \\
    A'^{*} =
    [(c  ,(k,u_k),u  ) :-> +]
    [(c' ,v_j    ,u_j) :-> +]_{j=1}^n
    [(c_j,v_j    ,u_j) :-> -]_{j=1}^n
  }{
    G,c,a^{*},A^{*},u \xrightarrow{\Create~k} G A'^{*},c',a^{*},A^{*}A'^{*},u
  }

  \Inferrule{
    \children(G,c) = \{\e_j\}_{j=1}^n \\
    A'^{*} = [\e_j :-> -]_{j=1}^n
  }{
    G,c,a^{*},A^{*},u \xrightarrow{\Destroy} G A'^{*},c,a^{*},A^{*}A'^{*},u
  }

  \Inferrule{
    u \text{ fresh} \\
    A = [(c,v,u) :-> +]
  }{
    G,c,a^{*},A^{*},u \xrightarrow{\Restore~v} G A',c,a^{*},A^{*}A,u
  }
\end{mathpar}

\subsection{Communication}
\label{sec:communication}

\begin{mathpar}
  \Inferrule{
    a \in \A^{edit} \\
    G,c,a^{*},A^{*},u -->a G',c',a^{*},A'^{*},u
  }{
    G,c,a^{*},A^{*},u -->a G',c',a^{*}a,A'^{*},u
  }
\end{mathpar}

\paragraph{Globally Known Graph Actions} ~

$\known(E_1 \cdots E_n) = \{ A_j^{*} :~ E_j = (G_j,c_j,a_j^{*},A_j^{*}), j = 1,\ldots,n \}$

\paragraph{Replay} ~

$\boxed{E \xrightarrow{a^{*}} E}$
%
\begin{mathpar}
  \Inferrule{
    E -->a E' \\
    E' \xrightarrow{a^{*}} E'' \\
  }{
    E \xrightarrow{aa^{*}} E''
  }
\end{mathpar}

\paragraph{Publish} ~

$\boxed{E^{*} -->a E^{*}}$
%
\begin{mathpar}
  \Inferrule{
    E_j = (G_j,c_j,a_j^{*},A_j^{*},u_j) \\
    E_j \xrightarrow{a'^{*}} (G_j',c_j',a_j'^{*}, A_j'^{*},u_j) \\
    E_j' =
    ( G_j'
    , c_j'
    , a_j'^{*} \setminus a'^{*}
    , A_j'^{*} \setminus \known(E_1 \cdots E_n),u_j) \\
    j = 1,\ldots,n
  }{
    E_1 \cdots E_n \xrightarrow{\Send~ a'^{*} } E_1' \cdots E_n'
  }
\end{mathpar}

\end{document}
